\subsubsection{Mô tả bài toán}

Viết hàm \texttt{vmprint(pagetable\_t)} để in nội dung bảng trang trong RISC-V theo định dạng phân cấp. Mỗi cấp được thể hiện bằng dấu \texttt{..}, in ra chỉ số PTE, giá trị PTE (bao gồm các bit quyền truy cập), và địa chỉ vật lý của các PTE hợp lệ. Chèn lệnh gọi \texttt{vmprint()} vào \texttt{exec.c} để in bảng trang của tiến trình init (PID = 1).

\subsubsection{Phương pháp thực hiện}
\begin{enumerate}
\item Trong \texttt{kernel/vm.c}, viết hàm đệ quy \texttt{vmprintlevel()} để duyệt qua các cấp của bảng trang.
\item Viết hàm \texttt{vmprint()} để in địa chỉ bảng trang và gọi hàm đệ quy.
\item Trong \texttt{kernel/defs.h}, khai báo prototype của hàm \texttt{vmprint()}.
\item Trong \texttt{kernel/exec.c}, chèn lệnh gọi \texttt{vmprint()} cho tiến trình có PID = 1.
\end{enumerate}

\paragraph{Duyệt qua bảng trang}
Trong hàm \texttt{vmprintlevel()}:
\begin{itemize}
\item Duyệt qua 512 PTE trong bảng trang hiện tại.
\item Chỉ xử lý các PTE hợp lệ (có bit \texttt{PTE\_V} được set).
\item In ra độ sâu (số lượng \texttt{..}), chỉ số PTE, giá trị PTE, và địa chỉ vật lý.
\item Phân biệt PTE là bảng trang con (không có bit R|W|X) hay là trang lá (có ít nhất một bit R|W|X).
\item Nếu PTE trỏ đến bảng trang con, đệ quy xuống cấp tiếp theo với độ sâu tăng thêm 1.
\end{itemize}

\paragraph{In bảng trang}
Trong hàm \texttt{vmprint()}:
\begin{itemize}
\item In dòng đầu tiên chứa địa chỉ của bảng trang: \texttt{"page table 0x..."}.
\item Gọi \texttt{vmprintlevel()} với tham số bảng trang và độ sâu ban đầu là 0.
\item Sử dụng macro \texttt{PTE2PA()} từ \texttt{kernel/riscv.h} để chuyển đổi PTE thành địa chỉ vật lý.
\end{itemize}

\paragraph{Khai báo và sử dụng}
Trong hàm \texttt{kernel/defs.h}:
\begin{itemize}
\item Thêm dòng khai báo \texttt{void vmprint(pagetable\_t);} vào phần khai báo các hàm của \texttt{vm.c}.
\end{itemize}

Trong hàm \texttt{exec()} của \texttt{kernel/exec.c}:
\begin{itemize}
\item Sau khi gán \texttt{p->pagetable = pagetable} và trước câu lệnh \texttt{return argc}.
\item Thêm điều kiện kiểm tra: \texttt{if(p->pid == 1) vmprint(p->pagetable)}.
\item Điều này đảm bảo chỉ in bảng trang của tiến trình init (tiến trình đầu tiên).
\end{itemize}

\subsubsection{Cấu trúc bảng trang và logic ánh xạ}

\paragraph{Cấu trúc bảng trang phân cấp trong RISC-V}
Hệ thống RISC-V sử dụng bảng trang phân cấp 3 cấp (3-level page table) để ánh xạ địa chỉ ảo 39-bit sang địa chỉ vật lý. Mỗi cấp bảng trang chứa 512 mục (Page Table Entry - PTE), mỗi PTE có kích thước 64-bit.

Cấu trúc địa chỉ ảo 39-bit được chia thành các phần:
\begin{itemize}
\item \textbf{Bits 38-30 (9 bits):} Chỉ số vào bảng trang cấp 2 (L2) - 512 mục
\item \textbf{Bits 29-21 (9 bits):} Chỉ số vào bảng trang cấp 1 (L1) - 512 mục
\item \textbf{Bits 20-12 (9 bits):} Chỉ số vào bảng trang cấp 0 (L0) - 512 mục
\item \textbf{Bits 11-0 (12 bits):} Offset trong trang vật lý - 4096 bytes
\end{itemize}

\paragraph{Logic ánh xạ địa chỉ}
Quá trình dịch địa chỉ ảo sang địa chỉ vật lý diễn ra như sau:
\begin{enumerate}
\item \textbf{Bắt đầu từ bảng trang gốc (root page table):} Địa chỉ bảng trang gốc được lưu trong thanh ghi \texttt{satp}.
\item \textbf{Cấp 2 (L2):} Sử dụng bits 38-30 của địa chỉ ảo làm chỉ số để tra cứu PTE tương ứng trong bảng trang cấp 2.
\item \textbf{Kiểm tra PTE:}
\begin{itemize}
    \item Nếu PTE không hợp lệ (bit \texttt{V} = 0): Lỗi page fault.
    \item Nếu PTE là trang lá (có ít nhất một bit R, W, hoặc X = 1): Lấy địa chỉ vật lý từ PPN (Physical Page Number) và kết hợp với offset.
    \item Nếu PTE là bảng trang con (R = W = X = 0): Chuyển sang cấp tiếp theo.
\end{itemize}
\item \textbf{Cấp 1 (L1):} Nếu PTE ở cấp 2 là bảng trang con, sử dụng PPN để tìm bảng trang cấp 1, sau đó dùng bits 29-21 làm chỉ số tra cứu PTE.
\item \textbf{Cấp 0 (L0):} Tương tự, nếu PTE ở cấp 1 là bảng trang con, tiếp tục với bits 20-12.
\item \textbf{Tính địa chỉ vật lý cuối cùng:} Khi tìm được trang lá, địa chỉ vật lý = (PPN << 12) | offset (12 bits cuối của địa chỉ ảo).
\end{enumerate}

\paragraph{Trang lá (Leaf Page) và bảng trang con}
\textbf{Trang lá} là PTE cuối cùng trong chuỗi tra cứu, trỏ đến trang vật lý thực sự chứa dữ liệu. Đặc điểm nhận biết trang lá:
\begin{itemize}
\item Có ít nhất một bit trong R (Read), W (Write), hoặc X (Execute) được set = 1.
\item PPN của trang lá chứa địa chỉ vật lý của trang dữ liệu (data page).
\item Trang lá có thể xuất hiện ở bất kỳ cấp nào (L2, L1, hoặc L0), cho phép superpage (trang lớn hơn 4KB).
\end{itemize}

\textbf{Bảng trang con} (intermediate page table) là PTE trỏ đến bảng trang ở cấp thấp hơn. Đặc điểm:
\begin{itemize}
\item Không có bit R, W, X nào được set (R = W = X = 0).
\item PPN của bảng trang con chứa địa chỉ vật lý của bảng trang cấp tiếp theo.
\item Chỉ xuất hiện ở cấp 2 (L2) hoặc cấp 1 (L1), không ở cấp 0 (L0).
\end{itemize}

\paragraph{Quyền truy cập (Permission Bits)}
Mỗi PTE chứa các bit quyền truy cập quan trọng:
\begin{itemize}
\item \textbf{V (Valid, bit 0):} PTE có hợp lệ hay không. Nếu V = 0, mọi truy cập đều gây page fault.
\item \textbf{R (Read, bit 1):} Cho phép đọc. Nếu R = 1, trang có thể được đọc.
\item \textbf{W (Write, bit 2):} Cho phép ghi. Nếu W = 1, trang có thể được ghi. Thường W = 1 kéo theo R = 1.
\item \textbf{X (Execute, bit 3):} Cho phép thực thi. Nếu X = 1, trang có thể chứa mã lệnh để thực thi.
\item \textbf{U (User, bit 4):} Cho phép user mode truy cập. Nếu U = 1, trang có thể được truy cập từ user mode, nếu U = 0, chỉ kernel mode mới truy cập được.
\item \textbf{G (Global, bit 5):} Ánh xạ toàn cục, không bị xóa khi chuyển ngữ cảnh.
\item \textbf{A (Accessed, bit 6):} Đánh dấu trang đã được truy cập.
\item \textbf{D (Dirty, bit 7):} Đánh dấu trang đã được ghi.
\end{itemize}

Các tổ hợp quyền thường gặp:
\begin{itemize}
\item \texttt{R=1, W=0, X=0, U=1:} Trang chỉ đọc của user (read-only data).
\item \texttt{R=1, W=1, X=0, U=1:} Trang đọc-ghi của user (data, heap, stack).
\item \texttt{R=1, W=0, X=1, U=1:} Trang mã lệnh của user (code/text segment).
\item \texttt{R=1, W=1, X=0, U=0:} Trang đọc-ghi của kernel.
\item \texttt{R=0, W=0, X=0, V=1:} Bảng trang con (không phải trang lá).
\end{itemize}

\subsubsection{Phân tích output của vmprint()}
Khi chạy hàm \texttt{vmprint()}, output có dạng:
\begin{verbatim}
page table 0x0000000087f6e000
..0: pte 0x0000000021fda801 pa 0x0000000087f6a000
.. ..0: pte 0x0000000021fda401 pa 0x0000000087f69000
.. .. ..0: pte 0x0000000021fdac1f pa 0x0000000087f6b000
.. .. ..1: pte 0x0000000021fda00f pa 0x0000000087f68000
\end{verbatim}

Giải thích từng dòng:
\begin{itemize}
\item \textbf{Dòng 1:} \texttt{page table 0x0000000087f6e000} - Địa chỉ vật lý của bảng trang gốc (root page table).
\item \textbf{Dòng 2:} \texttt{..0: pte 0x0000000021fda801 pa 0x0000000087f6a000}
\begin{itemize}
    \item \texttt{..0}: Cấp 2 (L2), PTE tại chỉ số 0.
    \item \texttt{pte 0x0000000021fda801}: Giá trị PTE. Bit 0 (V) = 1 (hợp lệ), các bit R|W|X = 0 → Đây là bảng trang con.
    \item \texttt{pa 0x0000000087f6a000}: Địa chỉ vật lý của bảng trang cấp 1.
\end{itemize}
\item \textbf{Dòng 3:} \texttt{.. ..0: pte 0x0000000021fda401 pa 0x0000000087f69000}
\begin{itemize}
    \item \texttt{.. ..0}: Cấp 1 (L1), PTE tại chỉ số 0.
    \item Giá trị PTE cho thấy đây cũng là bảng trang con (R|W|X = 0).
    \item Trỏ đến bảng trang cấp 0 tại địa chỉ \texttt{0x87f69000}.
\end{itemize}
\item \textbf{Dòng 4:} \texttt{.. .. ..0: pte 0x0000000021fdac1f pa 0x0000000087f6b000}
\begin{itemize}
    \item \texttt{.. .. ..0}: Cấp 0 (L0), PTE tại chỉ số 0.
    \item \texttt{pte 0x0000000021fdac1f}: Giá trị PTE kết thúc bằng \texttt{0x1f} (binary: 11111) → V=1, R=1, W=1, X=1, U=1.
    \item Đây là \textbf{trang lá} với quyền đọc, ghi, thực thi cho user mode.
    \item \texttt{pa 0x0000000087f6b000}: Địa chỉ vật lý của trang dữ liệu.
\end{itemize}
\item \textbf{Dòng 5:} \texttt{.. .. ..1: pte 0x0000000021fda00f pa 0x0000000087f68000}
\begin{itemize}
    \item \texttt{.. .. ..1}: Cấp 0 (L0), PTE tại chỉ số 1.
    \item \texttt{pte 0x0000000021fda00f}: Kết thúc bằng \texttt{0x0f} (binary: 01111) → V=1, R=1, W=1, X=1, U=0.
    \item Đây là \textbf{trang lá} với quyền đọc, ghi, thực thi nhưng chỉ cho kernel mode (U=0).
    \item Địa chỉ vật lý: \texttt{0x87f68000}.
\end{itemize}
\end{itemize}

Từ output này, ta thấy rõ cấu trúc phân cấp: Bảng trang gốc (L2) → Bảng trang con (L1) → Bảng trang con (L0) → Các trang lá chứa dữ liệu thực sự. Mỗi cấp được thể hiện bằng số lượng dấu \texttt{..}, và quyền truy cập được mã hóa trong các bit cuối của giá trị PTE.